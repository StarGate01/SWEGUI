\documentclass[shortpres,usenames,dvipsnames]{beamer}
\usetheme{CambridgeUS}

% Depending on build configuration, one of these packages will
% enable unicode
\usepackage[utf8]{inputenc}
\usepackage{fontspec}
\usepackage{listings}

%Images
\usepackage{graphics}
\usepackage{graphicx, svg}
\usepackage{caption}
\usepackage{adjustbox}

%Mixed
\usepackage{subfig}
\usepackage{multicol}
\usepackage{xcolor}
\usepackage{colortbl}
\usepackage{pgfplots}
\usepackage{xmpmulti}
\usepackage{verbatim}
\usepackage{array}
\usepackage{tabularx}
\usepackage{cprotect}

%Tikz
\usepackage{tikz}
\usepackage{environ}
\usetikzlibrary{positioning}



\usepackage{algorithm,algpseudocode}  %for algorithm environmenstacle in the bathymetry to show the effect of the soruce terms in 2D.t

\setbeamertemplate{footline}
{
	\leavevmode%
	\hbox{%
		\begin{beamercolorbox}[wd=.333333\paperwidth,ht=2.25ex,dp=1ex,center]{author in head/foot}%
			\usebeamerfont{author in head/foot}\insertshortauthor%~~\beamer@ifempty{\insertshortinstitute}{}{(\insertshortinstitute)}
		\end{beamercolorbox}%
		\begin{beamercolorbox}[wd=.333333\paperwidth,ht=2.25ex,dp=1ex,center]{title in head/foot}%
			\usebeamerfont{title in head/foot}\insertshorttitle
		\end{beamercolorbox}%
		\begin{beamercolorbox}[wd=.333333\paperwidth,ht=2.25ex,dp=1ex,right]{date in head/foot}%
			\usebeamerfont{date in head/foot}\insertshortdate{}\hspace*{2em}
			\insertframenumber{} / \inserttotalframenumber\hspace*{2ex}
		\end{beamercolorbox}}%
		\vskip0pt%
	}\part{title}
	\beamertemplatenavigationsymbolsempty
	
	
	%color specification---------------------------------------------------------------
	\definecolor{TUMblue}{rgb}{0.00, 0.40, 0.74}
	\definecolor{TUMgray}{rgb}{0.85, 0.85, 0.86}
	\definecolor{TUMpantone285C}{rgb}{0.00, 0.45, 0.81}
	\definecolor{lightblue}{rgb}{0.7529,0.8118,0.9333}
	
	\setbeamercolor{block title}{fg=white, bg=TUMpantone285C}
	\setbeamercolor{block body}{bg=lightblue}
	\setbeamertemplate{blocks}[rounded][shadow=true]
	%----------------------------------------------------------------------------------
	
	\setbeamercolor{frametitle}{fg=TUMblue, bg=white}
	\setbeamercolor{palette primary}{fg=TUMblue,bg=TUMgray}
	\setbeamercolor{palette secondary}{use=palette primary,fg=TUMblue,bg=white}
	\setbeamercolor{palette tertiary}{use=palette primary,fg=white, bg=TUMblue}
	\setbeamercolor{palette quaternary}{use=palette primary,fg=white,bg=TUMpantone285C}
	
	
	\setbeamercolor{title}{bg=white,fg=TUMblue}
	\setbeamercolor{item projected}{use=item,fg=black,bg = lightblue}
	\setbeamercolor{block title}{fg=black, bg=lightblue}
	\setbeamercolor{block body}{bg=white}
	\setbeamertemplate{blocks}[rounded][shadow=true]
	%----------------------------------------------------------------------------------
	
	
	%############### Self defined commands ##############################
	\newcommand{\imgvoffset}{-20pt}
	\newcommand{\texthoffset}{20pt}
	\newcommand{\imgfullscale}{0.75}
	\newcommand{\imgcolscale}{0.9}
	
	\captionsetup[subfigure]{labelformat=empty}		%Disable enumeration of subfigures
	%####################################################################
	
	%############### Title information ###############
	\title[{Tsunami simulation}]{Project}
	\author[Bellamy, Honal, Wieser]{Gruppe 03\\George Bellamy, Christoph Honal, Felix Wieser\\\vspace{10pt}{\small Bachelorpraktikum}}
	\institute[TU M\"unchen]{Technical University of Munich}
	\date{9. Januar 2018}
	%#################################################
	
	%############### Tikz picture configuration ###############
	\newcommand{\pgfglobalscale}{0.7}
	\newcommand{\pgfglobalheadervspace}{\vspace{5pt}\\}
	%##########################################################
	
	\lstset{escapeinside={<@}{@>}}
	
	\begin{document}
		\maketitle
		
%TODO change pictures		
		
\begin{frame}{Overview}
	\begin{figure}
		\subfloat[User Interface \tiny **]{\includegraphics[clip,width=0.3\linewidth]
			{img/UI_old.jpg}}
		\hspace{20pt}
		\subfloat[Backend \tiny *]{\includegraphics[clip,width=0.3\linewidth]
			{img/oszi.jpg}}
		\hspace{40pt}\vspace{10pt}\\
		\subfloat[Demonstration]{\includegraphics[clip,width=0.3\linewidth]
			{img/tuhoku_mixed.png}}
		\vfill
		\flushleft
		{\fontsize{5}{5} \selectfont *Image: \url{https://commons.wikimedia.org/wiki/File:1tt.jpg}}
		\flushleft
		{\fontsize{5}{5} \selectfont **Image: \url{https://upload.wikimedia.org/wikipedia/commons/e/e8/NS_Savannah_-_Reactor_Control_Panel_-_SCRAM_Button.jpg}}
	\end{figure}
\end{frame}

\begin{frame}[fragile]{UI}
	\begin{figure}
		\includegraphics[clip, width=\linewidth]{img/dummy_image.jpg}
	\end{figure}
\end{frame}

\begin{frame}[fragile]{Toolbar}
	\begin{figure}
		\includegraphics[clip, width=80mm]{img/dummy_image.jpg}
	\end{figure}
	\begin{itemize}
		\item Open new files (NetCDF)
		\item Control the simulation time and speed
		\item Change the displayed section
		\item Screen shots can be exported
	\end{itemize}
\end{frame}

\begin{frame}[fragile]{Rendered view}
	\begin{figure}
		\includegraphics[clip, width=80mm]{img/dummy_image.jpg}
	\end{figure}
	\begin{itemize}
		\item Color shading is chosen according to the max/min values
		\item Selected points are highlighted 
	\end{itemize}
\end{frame}

\begin{frame}[fragile]{Probes}
	\begin{figure}
		\includegraphics[clip, width=80mm]{img/Buoy.jpg}
	\end{figure}
	\begin{itemize}
		\item Select a point in the rendered view and it is saved on the sidebar
		\item Points can be selected and their information is shown
	\end{itemize}
	\vfill
	\flushleft
	{\fontsize{5}{5} \selectfont \url{http://axystechnologies.com/wp-content/uploads/2013/09/HydroLevel-Mini-Buoy-IT.jpg}}
\end{frame}

\begin{frame}[fragile]{Graphs}
	\begin{figure}
		\includegraphics[clip, width=80mm]{img/dummy_image.jpg}
	\end{figure}
	\begin{itemize}
		\item Click two points on the rendered view and the cross section between these two is selected
		\item A separate window shows the cross section graph
		\item Screen shots of these can be exported
	\end{itemize}
\end{frame}

\begin{frame}[fragile]{Shader implementation}
	\textbf{Overview}
	\begin{itemize}
		\item Stuff
	\end{itemize}
\end{frame}

\begin{frame}[fragile]{UI implementation}
	\begin{figure}
		\includegraphics[clip, width=20mm]{img/GTK_logo.png}
	\end{figure}
	\textbf{gtkmm for C++}
	\begin{itemize}
		\item Easy to use library 
		\item Based on gtk UI, used in gimp originally
		\item Rendered view is done by shader, all UI is gtkmm
		\item Toolbar icons and interface are integrated in to gtkmm 
	\end{itemize}
	\vfill
	\flushleft
	{\fontsize{5}{5} \selectfont \url{https://de.wikipedia.org/wiki/Gtkmm#/media/File:GTK2B_logo.svg}}
\end{frame}
	
\begin{frame}[fragile]{UI implementation}
	\begin{figure}
		\includegraphics[clip, width=75mm]{img/glade.png}
	\end{figure}
	\textbf{glade}\\
	\begin{itemize}
		\item WYSIWYG interface editor 
		\item Generates an XML which describes the UI
		\item The UI is built with an extra widget for the renderer
	\end{itemize}
\end{frame}

\begin{frame}[fragile]{UI implementation}
	\begin{figure}
		\includegraphics[clip, width=50mm]{img/sfml.png}
	\end{figure}
	\textbf{SFML}\\
	\begin{itemize}
		\item openGL interface
		\item Rendering view uses openGL to display data
	\end{itemize}
	\vfill
	\flushleft
	{\fontsize{5}{5} \selectfont \url{https://upload.wikimedia.org/wikipedia/commons/thumb/b/bf/SFML2.svg/230px-SFML2.svg.png}}
\end{frame}


\begin{frame}[fragile]{Code Structure}
bla
\end{frame}

\begin{frame}[fragile]{Demonstration}
	\begin{figure}
		\includegraphics[clip, width=\linewidth]{img/demo.jpeg}
	\end{figure}
	
	\vfill
	\flushleft
	{\fontsize{5}{5} \selectfont \url{https://news.uns.purdue.edu/images/2012/chemistry-show.jpg}}
\end{frame}

\begin{frame}{}
	\begin{figure}
		\includegraphics[clip, width=\imgfullscale\linewidth]{img/tsunami.jpg}
	\end{figure}
	\centering
	\vspace{10pt}
	Thank you for your attention
	\\
	\vfill
	\flushleft
	{\fontsize{5}{5} \selectfont \url{http://userscontent2.emaze.com/images/88c09d66-4283-49c0-9f80-9eb8fd05e30f/16101782-ea98-4b06-b114-4637be705926.jpg}}
\end{frame}
\end{document}
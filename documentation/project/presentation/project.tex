\documentclass[shortpres,usenames,dvipsnames]{beamer}
\usetheme{CambridgeUS}

% Depending on build configuration, one of these packages will
% enable unicode
\usepackage[utf8]{inputenc}
\usepackage{fontspec}
\usepackage{listings}

%Images
\usepackage{graphics}
\usepackage{graphicx, svg}
\usepackage{caption}
\usepackage{adjustbox}

%Mixed
\usepackage{subfig}
\usepackage{multicol}
\usepackage{xcolor}
\usepackage{colortbl}
\usepackage{pgfplots}
\usepackage{xmpmulti}
\usepackage{verbatim}
\usepackage{array}
\usepackage{tabularx}
\usepackage{cprotect}

%Tikz
\usepackage{tikz}
\usepackage{environ}
\usetikzlibrary{positioning}



\usepackage{algorithm,algpseudocode}  %for algorithm environmenstacle in the bathymetry to show the effect of the soruce terms in 2D.t

\setbeamertemplate{footline}
{
	\leavevmode%
	\hbox{%
		\begin{beamercolorbox}[wd=.333333\paperwidth,ht=2.25ex,dp=1ex,center]{author in head/foot}%
			\usebeamerfont{author in head/foot}\insertshortauthor%~~\beamer@ifempty{\insertshortinstitute}{}{(\insertshortinstitute)}
		\end{beamercolorbox}%
		\begin{beamercolorbox}[wd=.333333\paperwidth,ht=2.25ex,dp=1ex,center]{title in head/foot}%
			\usebeamerfont{title in head/foot}\insertshorttitle
		\end{beamercolorbox}%
		\begin{beamercolorbox}[wd=.333333\paperwidth,ht=2.25ex,dp=1ex,right]{date in head/foot}%
			\usebeamerfont{date in head/foot}\insertshortdate{}\hspace*{2em}
			\insertframenumber{} / \inserttotalframenumber\hspace*{2ex}
		\end{beamercolorbox}}%
		\vskip0pt%
	}\part{title}
	\beamertemplatenavigationsymbolsempty
	
	
	%color specification---------------------------------------------------------------
	\definecolor{TUMblue}{rgb}{0.00, 0.40, 0.74}
	\definecolor{TUMgray}{rgb}{0.85, 0.85, 0.86}
	\definecolor{TUMpantone285C}{rgb}{0.00, 0.45, 0.81}
	\definecolor{lightblue}{rgb}{0.7529,0.8118,0.9333}
	
	\setbeamercolor{block title}{fg=white, bg=TUMpantone285C}
	\setbeamercolor{block body}{bg=lightblue}
	\setbeamertemplate{blocks}[rounded][shadow=true]
	%----------------------------------------------------------------------------------
	
	\setbeamercolor{frametitle}{fg=TUMblue, bg=white}
	\setbeamercolor{palette primary}{fg=TUMblue,bg=TUMgray}
	\setbeamercolor{palette secondary}{use=palette primary,fg=TUMblue,bg=white}
	\setbeamercolor{palette tertiary}{use=palette primary,fg=white, bg=TUMblue}
	\setbeamercolor{palette quaternary}{use=palette primary,fg=white,bg=TUMpantone285C}
	
	
	\setbeamercolor{title}{bg=white,fg=TUMblue}
	\setbeamercolor{item projected}{use=item,fg=black,bg = lightblue}
	\setbeamercolor{block title}{fg=black, bg=lightblue}
	\setbeamercolor{block body}{bg=white}
	\setbeamertemplate{blocks}[rounded][shadow=true]
	%----------------------------------------------------------------------------------
	
	
	%############### Self defined commands ##############################
	\newcommand{\imgvoffset}{-20pt}
	\newcommand{\texthoffset}{20pt}
	\newcommand{\imgfullscale}{0.75}
	\newcommand{\imgcolscale}{0.9}
	
	\captionsetup[subfigure]{labelformat=empty}		%Disable enumeration of subfigures
	%####################################################################
	
	%############### Title information ###############
	\title[{Tsunami simulation}]{Project}
	\author[Bellamy, Honal, Wieser]{Gruppe 03\\George Bellamy, Christoph Honal, Felix Wieser\\\vspace{10pt}{\small Bachelorpraktikum}}
	\institute[TU M\"unchen]{Technical University of Munich}
	\date{30. Januar 2018}
	%#################################################
	
	%############### Tikz picture configuration ###############
	\newcommand{\pgfglobalscale}{0.7}
	\newcommand{\pgfglobalheadervspace}{\vspace{5pt}\\}
	%##########################################################
	
	\lstset{escapeinside={<@}{@>}}
	
	\begin{document}
		\maketitle
		
		
\begin{frame}{Overview}
	\begin{figure}
		\subfloat[User Interface \tiny **]{\includegraphics[clip,width=0.3\linewidth]
			{img/UI_old.jpg}}
		\hspace{20pt}
		\subfloat[Backend \tiny *]{\includegraphics[clip,width=0.3\linewidth]
			{img/oszi.jpg}}
		\hspace{40pt}\vspace{10pt}\\
		\subfloat[Demonstration]{\includegraphics[clip,width=0.3\linewidth]
			{img/sc_2.PNG}}
		\vfill
		\flushleft
		{\fontsize{5}{5} \selectfont *Image: \url{https://commons.wikimedia.org/wiki/File:1tt.jpg}}
		\flushleft
		{\fontsize{5}{5} \selectfont **Image: \url{https://upload.wikimedia.org/wikipedia/commons/e/e8/NS_Savannah_-_Reactor_Control_Panel_-_SCRAM_Button.jpg}}
	\end{figure}
\end{frame}

\begin{frame}[fragile]{UI}
	\begin{figure}
		\includegraphics[clip, width=\linewidth]{img/sc_3.PNG}
	\end{figure}
\end{frame}

\begin{frame}[fragile]{Toolbar}
	\begin{figure}
		\includegraphics[clip, width=60mm]{img/toolbar.png}
	\end{figure}
	\begin{itemize}
		\item Open new files (NetCDF)
		\item Control the simulation step
		\item Open layer selection
		\item Export screen shots
	\end{itemize}
\end{frame}

\begin{frame}[fragile]{Data visualisation}
	\begin{figure}
		\includegraphics[clip, width=60mm]{img/datavis.png}
	\end{figure}
	\begin{itemize}
		\item Color shading is chosen according to the max/min values
		\item Probes are shown as crosshaires
		\item Different data sets can be displayed
	\end{itemize}
\end{frame}

\begin{frame}[fragile]{Pan and Zoom}
	\begin{figure}
		\includegraphics[clip, width=60mm]{img/zoom.png}
	\end{figure}
	\begin{itemize}
		\item Pan the view by holding the mouse and dragging
		\item Zoom in and out with the mouse wheel
	\end{itemize}
\end{frame}

\begin{frame}[fragile]{Layer selection}
	\begin{figure}
		\includegraphics[clip, width=100mm]{img/sc_6.PNG}
	\end{figure}
	\begin{itemize}
		\item Select between data sets to display in color grades
		\item Clip displayed range to visualize small changes
		\item Choose what items to draw on the screen
	\end{itemize}
\end{frame}

\begin{frame}[fragile]{Probes}
	\begin{figure}t
		\includegraphics[clip, width=75mm]{img/Buoy.jpg}
	\end{figure}
	\begin{itemize}
		\item Select a point in the data visualization or create one with its coordinates
		\item Points can be selected and their information is shown
		\item Custom naming for probes
	\end{itemize}
	\flushleft
	{\fontsize{5}{5} \selectfont \url{http://axystechnologies.com/wp-content/uploads/2013/09/HydroLevel-Mini-Buoy-IT.jpg}}
\end{frame}

\begin{frame}[fragile]{Data display}
\begin{columns}
\begin{column}{0.5\textwidth}
\begin{itemize}
		\item Displays probe data
		\newline
		\item Optional extra windows to monitor multiple probes
		\newline
		\item Dedicated values or graph over time with max/smin values and current time as vertical lines
		\newline
		\item Export graphs as images 
	\end{itemize}
\end{column}
\begin{column}{0.5\textwidth}
	\begin{figure}
		\includegraphics[clip, width=55mm]{img/sc_5.PNG}
	\end{figure}
\end{column}
\end{columns}
\end{frame}


\begin{frame}[fragile]{Code Structure}
\textbf{Three major components:}\\
\begin{itemize}
		\item GUI
			\begin{itemize}
			\item Glade preset xml
			\item Button and menu functions
			\end{itemize}
		\item Renderer
			\begin{itemize}
			\item Data visualization
			\item Information display top left
			\end{itemize}
		\item Probes
			\begin{itemize}
			\item Manage creation and data of probes
			\item Graphs and data output
			\end{itemize}
	\end{itemize}
\end{frame}


\begin{frame}[fragile]{UI implementation - toolset}
	\begin{figure}
		\includegraphics[clip, width=20mm]{img/GTK_logo.png}
	\end{figure}
	\textbf{gtkmm for C++}
	\begin{itemize}
		\item Easy to use library 
		\item Based on gtk UI, used in gimp originally
		\item Rendered view is done by shader, all UI is gtkmm
		\item Toolbar icons and interface are integrated in to gtkmm 
	\end{itemize}
	\vfill
	\flushleft
	{\fontsize{5}{5} \selectfont \url{https://de.wikipedia.org/wiki/Gtkmm#/media/File:GTK2B_logo.svg}}
\end{frame}
	
\begin{frame}[fragile]{UI implementation - toolset}
	\begin{figure}
		\includegraphics[clip, width=75mm]{img/glade.png}
	\end{figure}
	\textbf{glade}\\
	\begin{itemize}
		\item WYSIWYG interface editor 
		\item Generates an XML which describes the UI
		\item The UI is built with an extra widget for the renderer
	\end{itemize}
\end{frame}

\begin{frame}[fragile]{Graphs implementation - toolset}
	\begin{figure}
		\includegraphics[clip, width=60mm]{img/cairo.png}
	\end{figure}
	\textbf{Cairo and Pango}\\
	\begin{itemize}
		\item Probes graph drawing
		\item Cairo: 2D graphics library, integrates in to GTK
		\item Pango: Text layout toolset, integrates in to cairo
	\end{itemize}
\end{frame}

\begin{frame}[fragile]{Shader implementation - toolset}
	\begin{figure}
		\includegraphics[clip, width=50mm]{img/sfml.png}
	\end{figure}
	\textbf{SFML}\\
	\begin{itemize}
		\item openGL interface
		\item Rendering view uses openGL to display data
	\end{itemize}
	$\rightarrow$ \textbf{self implemented opengl pixelshader}\\
	\vfill
	\flushleft
	{\fontsize{5}{5} \selectfont \url{https://upload.wikimedia.org/wikipedia/commons/thumb/b/bf/SFML2.svg/230px-SFML2.svg.png}}
\end{frame}

\begin{frame}[fragile]{Shader implementation - transport encoding}
	\begin{figure}
		\includegraphics[clip, width=50mm]{img/transport_encoding.jpg}
	\end{figure}
	\begin{itemize}
		\item Encode float to RGB double for renderer
	\end{itemize}
\end{frame}


\begin{frame}[fragile]{Demonstration}
	\begin{figure}
		\includegraphics[clip, width=\linewidth]{img/demo.jpeg}
	\end{figure}
	
	\vfill
	\flushleft
	{\fontsize{5}{5} \selectfont \url{https://news.uns.purdue.edu/images/2012/chemistry-show.jpg}}
\end{frame}

\begin{frame}{}
	\begin{figure}
		\includegraphics[clip, width=\imgfullscale\linewidth]{img/tsunami.jpg}
	\end{figure}
	\centering
	\vspace{10pt}
	Thank you for your attention
	\\
	\vfill
	\flushleft
	{\fontsize{5}{5} \selectfont \url{http://userscontent2.emaze.com/images/88c09d66-4283-49c0-9f80-9eb8fd05e30f/16101782-ea98-4b06-b114-4637be705926.jpg}}
\end{frame}
\end{document}
\documentclass[paper=a4]{proc}
\usepackage[ngerman]{babel}

%Images
\usepackage{graphicx, svg}

%Tikz
\usepackage{tikz}
\usepackage{environ}
\usepackage{pgfplots}
\usepgfplotslibrary{fillbetween}
\usetikzlibrary{positioning}

\pagenumbering{gobble}		%Disable page numbers

\title{Graphical Data Visualization Tool - How To Use The Interface}
\author{George Bellamy, Christoph Honal, Felix Wieser}

\begin{document}
	\maketitle
	\thispagestyle{plain}	%Disable page numbers
	\section{Overview}
		This is a graphical user interface to interactively visualize the output of the SWE tsunami simulation and provide basic analysis functions. It takes pre-calculated data sets as NetCDF and displays these. 
		
Once you open the program you can see that the UI is divided in to different parts. At the top is the toolbar which can be used to open new files, control the simulation speed and open the layer view. Below this to the left is the data visualization where the simulation data can be observed in a colored 2D birds eye view. It can be double clicked to place probes which let you observe the data of a point in more detail via dedicated values or graphing this point over time.
To the top right is the data display where information on the currently selected probe is shown. Below this is a list of probe points witch can be added, changed or deleted.
		
	\section{UI elements}
	We will now further explain the different parts of the UI in more detail. 
	
		\subsection{Toolbar}
		It is located at the top of the GUI. Looking at the buttons from left to right we have:
		\begin{itemize}
		\item Open: This opens a NetCDF file in a selection window. The chosen file is imported and displayed
		\item Reset time: The simulation time is reset to the beginning.
		\item Simulation step: This is the current simulation step, it can be overridden by tipping a new number or increased or decreased to the right. The actual time is shown at the bottom of the screen.
		\item Play: The simulation is stepped through progressively
		\item Layer selection: This opens the layer selection in which the data visualization can be adjusted.
		\item Rendering options: Here the different items rendered in the data visualization can be turned off or on like probes, their names and indicators, coordinates and general info.
		\item Reset view: The view in the data visualization is reset to the original state: fully zoomed out.
		\item Screen shot: Export the data visualization as an png picture file.
		
		\end{itemize}
		
			
		\subsection{Data visualization}
		This fills most of the UI and is located on the lower left side. Here the data of the simulation is displayed with a color heightmap. The display can be altered via the layer selection. The color map is chosen automatically to cover the range from the lowest point to the highest point (per default). 
		
		The view can be panned and zoomed in or out. To change the currently viewed section of the simulation you can drag the view by holding the right or middle mouse button when above the data visualization and moving the mouse. To zoom in or out use the mouse wheel.

The epicenter of the tsunami is shown as pointed cones in a cross shape. On the top left there is information about the currently displayed visualization.
		
		Probe points are shown as cross hairs with arrows. New ones can be created by double-clicking a point or via the probes list. The currently selected probe is highlighted with a box around it. You can select one of the probes with a single left click on it. Then the values are displayed in the data display to the right. Below the data visualization you can find the current time of the simulation
		\subsection{Layer selection}
		This is opened via the tool bar. Here you can change what data is rendered. You can select between five different data sets to display by using the toggles to the right of the names of the data sets: bathymetry, water height, horizontal flux, vertical flux and total flux. 
		
		Further to the right of the corresponding data set you will find the clip function. The values you enter as minimum and maximum will clip the displayed data to within these boundaries. So any value larger or smaller will be represented in the color of the chosen maximum or minimum value. This is especially useful to monitor smaller changes more closely in the data visualization as small changes with large maximum and minimum values will not be well represented in the color grade.
		\subsection{Probes}
		These are points of interest selected on the data visualization. They are listed at the bottom right of the application. To create a new probe either double click on the rendered view or choose the add button at the bottom. To select an existing probe either click it in the list or in the rendered view. 
		
		The data of the selected probe will be displayed in the data display above the probe list. To edit a probe (including name or coordinates) or to delete it you can right click on it and select the corresponding function with a left click from the drop down menu. The point will stay at the selected coordinates even if you change the window size. 
		
		To monitor the data of more than one point you can right click on a point in the list and choose the display option from the drop down menu. A new window will open and allow you to monitor the probes data independently. You can close this window at any point, this will not delete the probe.
		\subsection{Data display}
		Above the probes on the right or in an extra window if opened by using the drop down menu of a probe. Here the data of the chosen probe is shown. The active probe can be changed by clicking on one in the list or in the rendered view. At the top you can select between displaying the current data as numbers or to graph the data from the probes point. The graph will draw the data from the probe point over the simulation time. A solid red line form the top to the bottom of the graph will indicate the current time of the simulation, a blue line in the graph will indicate the lowest and a pink line the highest value. To the left the values of these extrema will be displayed. To view the graph in more detail open the graph in an extra window by right clicking on the probe in the probe list and clicking open. Then you can resize the window to any size. The graph can also be exported as a png picture file by clicking on save.

\end{document}